\documentclass[12pt,a4paper]{article}
\usepackage[utf8]{inputenc}
\usepackage[russian]{babel}
\usepackage[left=2.0cm, top=3.0cm, right=2.0cm, bottom=2.5cm]{geometry}
\usepackage{indentfirst}
\sloppy
\usepackage{amsmath}
\usepackage{amsfonts}
\usepackage{amssymb}
\usepackage{graphicx}
\begin{document}
\textsc{Задача 7}

Первое начало термодинамики (= закон сохранения энергии):

$$ \Delta Q = \Delta U + A .$$

$ A = p \Delta V$, такая формула для работы (а не формула с интегралом) используется, потому что по условию предполагается, что изменения параметров малы. С этим мы разобрались на занятии.

Далее было непонятно, почему в решении
$$\Delta U = mc_{V}\Delta T.$$

(На самом деле, это объясняется в методичке, там где теория, но, увы, вчера я не заметил).

Причём эта формула справедлива для \underline{любого} процесса, какие могут встретиться в задачах (с идеальным газом). Почему это так?

Рассмотрим произвольный процесс 1 $\rightarrow$ 2.

Ещё рассмотрим произвольный изохорический процесс ($V = const$ $\Rightarrow$ $\Delta V=0$). По закону сохранения энергия (первое начало термодинамики) $\Delta Q_{iso\_V} = \Delta U_{iso\_V} + A_{iso\_V}$, но $\Delta V = 0$ $\Rightarrow$ $A_{iso\_V} = 0$. Тогда $\Delta Q_{iso\_V} = \Delta U_{iso\_V}$ и $mc_V\Delta T = \Delta U_{iso_V}$. $c_V$ --- удельная теплоёмкость при постоянном объёме (просто теплоёмкость в этом процессе; её можно было бы назвать и как-нибудь по-другому, изохорическая теплоёмкость, например). 

Т.е. справедливо, что $\Delta U_{iso\_V} = mc_V\Delta T$ в изохорическом процессе. 

То рассуждение с циклом, которое я пытался построить на занятии, действительно полезно, только следует рассматривать немного другой цикл. Давайте нарисуем его.

\begin{figure}[h!]
	\includegraphics[width=0.4\textwidth]{cycle.png}
\end{figure}

Через точку 2 проводим изотерму (iso\_T), а из точки 1 --- изохору (iso\_V) до пересечения с изотермой (точка 3). Заметим, что сделать такое построение можно единственным образом. Точки 1, 2 и 3 соединяет замкнутая кривая --- это цикл. Важная особенность цикла: пройдя по нему из точки 1 в точку 1, изменение внутренней энергии будет равно нулю: $\Delta U_{123} = 0$, потому что внутренняя энергия является функцией состояния и не зависит от способа перехода между состояниями. Тогда для цикла 1--2--3 имеем:

$$ \Delta U_{123} = \Delta U_{12} + \Delta U_{23} + \Delta U_{31} = 0.$$

$\Delta U_{12} = \Delta U$ (по определению), $\Delta U_{23} \sim \Delta T = 0 $ (т.к. идём по изотерме), $\Delta U_{31} = \Delta U_{iso\_V} = mc_V\Delta T_{13} = -mc_V\Delta T $ (см. выше).

Поэтому 

$$ \Delta U_{123} = \Delta U - mc_V\Delta T = 0.$$

Отсюда следует, что для любого перехода из состояния 1 в состояние 2 изменение внутренней энергии будет таким же, как для изохорического процесса, т.е. 
$$\Delta U = \Delta U_{iso\_V} = mc_V\Delta T.$$

Удачи. :)
\end{document}
