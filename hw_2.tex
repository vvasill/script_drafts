\documentclass[12pt,a4paper]{article}
\usepackage[utf8]{inputenc}
\usepackage[russian]{babel}
\usepackage[left=2.0cm, top=3.0cm, right=2.0cm, bottom=2.5cm]{geometry}
\usepackage{indentfirst}
\sloppy
\usepackage{amsmath}
\usepackage{amsfonts}
\usepackage{amssymb}
\usepackage{graphicx}
\begin{document}
\textbf{Первая неделя, занятие 2}
\vskip 5pt
\hrule
\vskip 15pt

\textcircled{1} Сколько м/c в $\frac{\text{мегаметры}}{\text{сутки}}$? Сколько световых лет в одном парсеке? Сколько микрометров в $10^{15}$ дециметров?

\textcircled{2} Велосипедист половину времени всего движения ехал со скоростью $v_1$, половину оставшегося пути со скоростью $v_2$, а последний участок шёл со скоростью $v_3$. Какова средняя скорость на всём пути?

\textcircled{3} Определить объём полости пробки стеклянного графина, если при погружении в воду она вытесняет 50 г воды и имеет массу 100 г. 

\textcircled{4} Медный шар имеет массу $m$ при объёме $V$. Сплошной этот шар или полый?

\textcircled{5} Стакан, заполненный до краёв маслом, имеет массу 214,6 г. Когда в этот стакан с водой поместили небольшой камень массой 29,8 г и часть масла вылилась наружу, масса стакана с содержимым оказалась равной 232 г. Определить плотность вещества камня.

\textcircled{6} Тело плотностью $\rho$ плавает на границе раздела двух жидкостей с плотностями $\rho_1$ и $\rho_2$. Какая часть объёма тела погружена в нижнюю жидкость? (см. рисунок)

\begin{figure}[h!]
	\includegraphics[width=0.2\textwidth]{img_1.png}
\end{figure}

\end{document}
