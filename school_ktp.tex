\documentclass[12pt,a4paper]{article}
\usepackage[utf8]{inputenc}
\usepackage[russian]{babel}
\usepackage{calc}
\usepackage[left=2.0cm, top=3.0cm, right=2.0cm, bottom=2.5cm, head = 50pt, foot = 50pt]{geometry}
\sloppy
\usepackage{amsmath}
\usepackage{amsfonts}
\usepackage{amssymb}
\usepackage{graphicx}
\usepackage{booktabs}
\usepackage{array}
\usepackage{ltablex}

\newlength\tempindent
\setlength{\tempindent}{\parindent}
\setlength{\parindent}{0pt}
\renewcommand{\indent}{\hspace*{\tempindent}}

\newlength\cellwidth
\setlength{\cellwidth}{0.9\linewidth/4}

\begin{document}

\begin{center}
\textsc{Календарно--тематическое планирование}\\
уроков физики 
\end{center}
\vskip -5 pt
\hrule
\vskip 10 pt

Класс: 9а

Учитель: Пушкарев Д.В.

Количество часов\\
\indent в год: 70\\
\indent в неделю: 2
  
Плановых контрольных работ: 7

Зачётов: 1

Плановых практических работ (лабораторных работ): 5

Литература\\
\indent основная:
Учебник Физика 10  Г.Я. Мякишев, Б.Б. Буховцев, Н.Н. Сотский, М.: Просвещение, 2014\\
\indent дополнительная: Сборник задач по физике 10--11кл. А.П. Рымкевич, М.: Дрофа, 2011
\vskip 10 pt
\textit{Планирование составлено на основе авторской программы В.С. Данюшенкова и О.В. Коршунова (сборник Программы общеобразовательных учреждений. Физика, астрономия 7--11 классы, М.: Просвещение, 2007)}

%\newcolumntype{P}[1]{>{\centering\arraybackslash}p{#1}}
\begin{tabularx}{\textwidth}{p{.08\linewidth}p{.6\linewidth}*{2}{p{.1\linewidth}}}
\toprule\addlinespace[1ex]
\centering{Номер урока} & \centering{Наименование раздела/темы} & \centering{Плановые сроки прохождения} & \centering{\hspace{0pt}Скорректированные сроки прохождения} \tabularnewline
\midrule
\endhead
1 & Физика и познание мира & 05.09 & 05.09 \tabularnewline
\multicolumn{4}{c}{\textbf{Кинематика (7 часов)}} \tabularnewline
2 & Основные понятия кинематики & 06.09 & \tabularnewline
3 & Скорость. Равномерное прямолинейное движение & 07.09 & 07.09\tabularnewline
4 & Относительность механического движения. Принцип относительности в механике & 12.09 & \tabularnewline
5 & Аналитическое описание равноускоренного прямолинейного движения  &  & \tabularnewline
6 & Свободное падение тел --- частный случай &  & \tabularnewline
7 & Равномерное движение точки по окружности &  & \tabularnewline
8 & К/р №1 "<Кинематика"> &  & \tabularnewline
\multicolumn{4}{c}{\textbf{Динамика и силы в природе (9 часов)}} \tabularnewline
9 & Масса и сила. Законы Ньютона, их экспериментальное подтверждение &  & \tabularnewline
10 & Решение задач на законы Ньютона &  & \tabularnewline
11 & Силы в механике. Гравитационные силы &  & \tabularnewline
12 & Cила тяжести и вес &  & \tabularnewline
13 & Решение задач по теме "<Гравитационные силы. Вес тела"> &  & \tabularnewline
14 & Силы упругости – силы электромагнитной природы &  & \tabularnewline
15 & Л/р №1 "<Изучение движения тела по окружности под действием силы упругости и силы тяжести">
 &  & \tabularnewline
16 & Силы трения &  & \tabularnewline
15 & К/р №2 "<Динамика. Силы в природе"> &  & \tabularnewline
17 &  &  & \tabularnewline
18 &  &  & \tabularnewline
19 &  &  & \tabularnewline
20 &  &  & \tabularnewline
21 &  &  & \tabularnewline
22 &  &  & \tabularnewline
23 &  &  & \tabularnewline
24 &  &  & \tabularnewline
25 &  &  & \tabularnewline
26 &  &  & \tabularnewline
27 &  &  & \tabularnewline
28 &  &  & \tabularnewline
29 &  &  & \tabularnewline
30 &  &  & \tabularnewline
31 &  &  & \tabularnewline
32 &  &  & \tabularnewline
33 &  &  & \tabularnewline
34 &  &  & \tabularnewline
33 &  &  & \tabularnewline
34 &  &  & \tabularnewline
33 &  &  & \tabularnewline
35 &  &  & \tabularnewline
36 &  &  & \tabularnewline
37 &  &  & \tabularnewline
\bottomrule
\end{tabularx}
\begin{flushright}
Всего часов: xx ч
\end{flushright}
\vskip20pt
Учитель: Пушкарев Д.В.
\vskip20pt
\textsc{согласовано}\\
Протокол заседания методического\\
объединения учителей 
\vskip20pt
\textsc{согласовано}\\
Зам. директора по УВР 

\end{document}
